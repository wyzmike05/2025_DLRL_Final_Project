\documentclass{beamer}
\usetheme{Madrid} % 你可以根据喜好更改主题,如 Boadilla, Copenhagen 等
\usecolortheme{default}

\usepackage[utf8]{inputenc}
\usepackage{amsmath, amssymb, amsfonts}
\usepackage{graphicx}
\usepackage{booktabs}

% --- 标题页信息 ---
\title[Conditional BFNs]{Conditional Generation of BFNs and its Advantages over Unconditional BFNs}
\date{Dec 24, 2025}

\begin{document}

% 1. 标题页
\begin{frame}
    \titlepage
\end{frame}

% 2. 目录页
\begin{frame}{Overview}
    \begin{enumerate}
        \item Introduction
        \item Framework of BFNs
        \item Experiments: Dynamically Binarized MNIST
    \end{enumerate}
\end{frame}

% 3. 介绍 - 文本
\begin{frame}{Introduction}
    BFNs is a novel generative model class that iteratively refines the parameters of independent distributions using Bayesian inference, guided by noisy data samples and neural network outputs. It combines the strengths of:
    \begin{itemize}
        \item \textbf{Bayesian inference}: Optimal information integration for individual variables.
        \item \textbf{Neural networks}: Contextual understanding across variables.
    \end{itemize}
    \vfill
    \textbf{Compared with Diffusion Models}: BFNs operate on distribution parameters rather than noisy data, ensuring continuity even for discrete data. This avoids the need for a predefined forward process.
    \vfill
    \textbf{Key idea}: Bayesian updates + neural networks for iterative refinement.
\end{frame}

% 4. 介绍 - 系统概览图
\begin{frame}{Introduction}
    \begin{figure}
        \centering
        \includegraphics[width=0.6\textwidth]{system_overview.png}
        \caption{Overview of Bayesian Flow Networks (from original paper)}
    \end{figure}
\end{frame}

% 5. 介绍 - 过程描述
\begin{frame}{Introduction}
    The figure represents one step of the modelling process of a Bayesian Flow Network. 
    \begin{itemize}
        \item The data is a ternary symbol sequence (e.g., 'B' and 'A').
        \item At each step, the network emits output distribution parameters based on previous input distribution.
        \item \textbf{Sender/Receiver distributions}: Continuous even for discrete data, created by adding noise.
        \item \textbf{Bayesian Update}: A sample from the sender is used to update input distribution parameters.
        \item \textbf{Loss function}: KL divergence from the receiver to the sender distribution (message from Alice to Bob).
    \end{itemize}
\end{frame}

% 6. 输入与发送者分布
\begin{frame}{Input and Sender Distributions}
    Given $D$-dimensional data $\mathbf{x} = (x^{(1)}, \dots, x^{(D)}) \in \mathcal{X}^D$, let $c \in \mathcal{C}$ be the conditional variable. Let
    $$\theta = (\theta^{(1)}, \dots, \theta^{(D)})$$
    be the parameters of a factorised \textbf{input distribution} $p_I(\cdot | \theta)$:
    \begin{equation}
        p_I(\mathbf{x} | \theta) = \prod_{d=1}^D p_I(x^{(d)} | \theta^{(d)})
    \end{equation}
    Let $p_S(\cdot | \mathbf{x}; \alpha)$ be a similarly factorized \textbf{sender distribution} with $\mathbf{y} = (y^{(1)}, \dots, y^{(D)}) \in \mathcal{Y}^D$:
    \begin{equation}
        p_S(\mathbf{y} | \mathbf{x}; \alpha) = \prod_{d=1}^D p_S(y^{(d)} | x^{(d)}; \alpha)
    \end{equation}
    where $\alpha \in \mathbb{R}^+$ is an accuracy parameter.
\end{frame}

% 7. 输出分布
\begin{frame}{Output Distributions $p_O(\cdot | \theta, c, t)$}
    Input parameters $\theta$, processing time $t \to$ a neural network $\Psi$.
    \vfill
    Output: $\Psi(\theta, c, t) = (\Psi^{(1)}(\theta, c, t), \dots, \Psi^{(D)}(\theta, c, t))$ to parameterize an \textbf{output distribution}:
    \begin{equation}
        p_O(\mathbf{x} | \theta, c, t) = \prod_{d=1}^D p_O(x^{(d)} | \Psi^{(d)}(\theta, c, t))
    \end{equation}
\end{frame}

% 8. 接收者分布
\begin{frame}{Receiver Distribution $p_R(\cdot | \theta, t, c, \alpha)$}
    Given sender distribution $p_S(\cdot | \mathbf{x}; \alpha)$ and output distribution $p_O(\cdot | \theta, c, t)$, the \textbf{receiver distribution} over $\mathcal{Y}^D$ is defined as:
    \begin{equation}
        p_R(\mathbf{y} | \theta, c; t, \alpha) = \mathbb{E}_{p_O(\mathbf{x}' | \theta, c; t)} [p_S(\mathbf{y} | \mathbf{x}'; \alpha)]
    \end{equation}
\end{frame}

% 9. 贝叶斯更新 1
\begin{frame}{Bayesian Updates}
    Given parameters $\theta$ and sender sample $\mathbf{y}$ drawn with accuracy $\alpha$, the \textbf{Bayesian update function} $h$ computes $\theta'$:
    \begin{equation}
        \theta' \gets h(\theta, \mathbf{y}, \alpha)
    \end{equation}
    The \textbf{Bayesian update distribution} $p_U(\cdot | \theta, \mathbf{x}; \alpha)$ is defined by marginalizing out $\mathbf{y}$:
    \begin{equation}
        p_U(\theta' | \theta, \mathbf{x}; \alpha) = \mathbb{E}_{p_S(\mathbf{y}|\mathbf{x};\alpha)} [\delta(\theta' - h(\theta, \mathbf{y}, \alpha))]
    \end{equation}
    where $\delta(\cdot - \mathbf{a})$ is the multivariate Dirac delta distribution.
\end{frame}

% 10. 贝叶斯更新 2
\begin{frame}{Bayesian Updates}
    Accuracy is additive: if $\alpha = \alpha_a + \alpha_b$, then
    \begin{equation}
        p_U(\theta'' | \theta, \mathbf{x}; \alpha) = \mathbb{E}_{p_U(\theta'|\theta,\mathbf{x};\alpha_a)} [p_U(\theta'' | \theta', \mathbf{x}; \alpha_b)]
    \end{equation}
    Probability of observing $\theta_n$ after $n$ samples with accuracies $\alpha_1, \dots, \alpha_n$:
    \begin{equation}
        \int \dots \int p_U(\theta_n | \theta_{n-1}, \mathbf{x}; \alpha_n) \dots p_U(\theta_1 | \theta_0, \mathbf{x}; \alpha_1) = p_U \left( \theta_n \bigg| \theta_0, \mathbf{x}; \sum_{i=1}^n \alpha_i \right)
    \end{equation}
\end{frame}

% 11. 精度计划
\begin{frame}{Accuracy Schedule $\beta(t)$}
    For continuous time $t \in [0, 1]$, let $\alpha(t) > 0$ be the accuracy rate. Define \textbf{accuracy schedule} $\beta(t)$:
    \begin{equation}
        \beta(t) = \int_0^t \alpha(t') dt'
    \end{equation}
    $\beta(t)$ is monotonically increasing, $\beta(0) = 0$, and:
    \begin{equation}
        \frac{d\beta(t)}{dt} = \alpha(t)
    \end{equation}
\end{frame}

% 12. 贝叶斯流分布
\begin{frame}{Bayesian Flow Distribution $p_F(\cdot | \mathbf{x}; t)$}
    The \textbf{Bayesian flow distribution} $p_F(\cdot | \mathbf{x}; t)$ is the marginal distribution over input parameters at time $t$:
    \begin{equation}
        p_F(\theta | \mathbf{x}; t) = p_U(\theta | \theta_0, \mathbf{x}; \beta(t))
    \end{equation}
\end{frame}

% 13. 损失函数 1
\begin{frame}{Loss Function $L(\mathbf{x}|c)$}
    Consider $n$ sender samples at times $t_i = i/n$. Accuracy at step $i$:
    \begin{equation}
        \alpha_i = \beta(t_i) - \beta(t_{i-1})
    \end{equation}
    Input parameter sequence is recursively updated:
    \begin{equation}
        \theta_i = h(\theta_{i-1}, \mathbf{y}, \alpha_i)
    \end{equation}
\end{frame}

% 14. 损失函数 2
\begin{frame}{Loss Function $L(\mathbf{x}|c)$}
    $n$-step discrete-time loss $L^n(\mathbf{x})$:
    \begin{equation}
        L^n(\mathbf{x}|c) = \mathbb{E}_{p(\theta_1, \dots, \theta_{n-1})} \sum_{i=1}^n D_{KL} (p_S(\cdot | \mathbf{x}; \alpha_i) \| p_R(\cdot | \theta_{i-1}, c; t_{i-1}, \alpha_i))
    \end{equation}
    where
    \begin{equation}
        p(\theta_1, \dots, \theta_n) = \prod_{i=1}^n p_U(\theta_i | \theta_{i-1}, \mathbf{x}; \alpha_i)
    \end{equation}
\end{frame}

% 15. 损失函数 3
\begin{frame}{Loss Function $L(\mathbf{x}|c)$}
    \textbf{Reconstruction loss}:
    \begin{equation}
        L^r(\mathbf{x}|c) = -\mathbb{E}_{p_F(\theta|\mathbf{x}, 1)} \ln p_O(\mathbf{x} | \theta, c; 1)
    \end{equation}
    \textbf{Total loss}:
    \begin{equation}
        L(\mathbf{x}|c) = L^n(\mathbf{x}|c) + L^r(\mathbf{x}|c)
    \end{equation}
\end{frame}

% 16. 离散时间损失计算技巧
\begin{frame}{A Trick in Computing Discrete-Time Loss $L^n(\mathbf{x}|c)$}
    Equation (14) can be rewritten using expectations over $i \sim U\{1, n\}$ and $\theta \sim p_F$:
    \begin{equation}
        L^n(\mathbf{x}|c) = n \mathbb{E}_{i \sim U\{1, n\}, p_F(\theta|\mathbf{x}; t_{i-1})} D_{KL} (p_S(\cdot | \mathbf{x}; \alpha_i) \| p_R(\cdot | \theta, c; t_{i-1}, \alpha_i))
    \end{equation}
    This allows \textbf{Monte-Carlo sampling} without computing the full $n$-step sum.
\end{frame}

% 17. 连续时间损失
\begin{frame}{Continuous-Time Loss $L^\infty(\mathbf{x})$}
    As $n \to \infty$, let $\epsilon = 1/n$:
    \begin{equation}
        L^\infty(\mathbf{x}) = \lim_{\epsilon \to 0} \frac{1}{\epsilon} \mathbb{E}_{t \sim U(\epsilon, 1), p_F(\theta|\mathbf{x}, t-\epsilon)} D_{KL} (p_S \| p_R)
    \end{equation}
    This removes noisy gradients and the need to fix $n$ during training.
\end{frame}

% 18. 连续时间损失 Prop 1
\begin{frame}{Continuous-Time Loss $L^\infty(\mathbf{x})$}
    For many sender-receiver pairs:
    \begin{equation}
        D_{KL} = \sum_{d=1}^D D_{KL} \left( \mathcal{N}(g(x^{(d)}), C\alpha^{-1}) \| P^{(d)}(\theta, t) * \mathcal{N}(0, C\alpha^{-1}) \right)
    \end{equation}
    \textbf{Prop 1}: For continuous $P$ with finite mean/variance, $P * \mathcal{N}(0, \sigma^2) \to \mathcal{N}(E[P], \sigma^2)$ as $\sigma^2 \to \infty$.
\end{frame}

% 20. 无条件生成结果
\begin{frame}{MNIST Sample Generation: Unconditional vs. Conditional BFN}
    \begin{figure}
        \centering
        \includegraphics[width=0.5\textwidth]{mnist_unconditional_samples.png}
        \caption{Samples in original paper (10,000 steps)}
    \end{figure}
\end{frame}

% 21. 条件生成结果
\begin{frame}{Conditional BFN (100 steps)}
    \begin{columns}
        \column{0.33\textwidth}
        \begin{figure}
            \centering
            \includegraphics[width=0.9\textwidth]{label_1_100_steps_mnist.png}
            \caption{1}
        \end{figure}
        \column{0.33\textwidth}
        \begin{figure}
            \centering
            \includegraphics[width=0.9\textwidth]{label_5_100_steps_mnist.png}
            \caption{5}
        \end{figure}
        \column{0.33\textwidth}
        \begin{figure}
            \centering
            \includegraphics[width=0.9\textwidth]{label_9_100_steps_mnist.png}
            \caption{9}
        \end{figure}
    \end{columns}
\end{frame}

% 22. 输入输出分布变化 1
\begin{frame}{MNIST Input and output distributions}
    \begin{figure}
        \centering
        \includegraphics[width=\textwidth]{mnist_input_output_distributions.png}
        \caption{Input and output distributions from original paper's unconditional BFN for a test image}
    \end{figure}
\end{frame}

% 23. 输入输出分布变化 2 (条件)
\begin{frame}{MNIST Input and output distributions}
    \begin{figure}
        \centering
        \begin{tabular}{cc}
            \includegraphics[width=0.45\textwidth]{input_dist_img0.png} &
            \includegraphics[width=0.45\textwidth]{output_dist_img0.png}
        \end{tabular}
        \caption{Input and output distributions from our conditionally trained BFN for a test image with a fixed label}
    \end{figure}
\end{frame}

% 23.5
\begin{frame}{Conditional Generative Models: Setup}
\textbf{Spaces and Measures:}
\begin{itemize}
    \item Data space $(\mathcal{X}, \mathcal{F}_X)$, condition space $(\mathcal{C}, \mathcal{F}_C)$
    \item Joint measure $\mu_{X,C}$ with marginal $\mu_C$ and conditional measures $\mu_{X|C=c}$
\end{itemize}

\textbf{Model Families:}
\begin{itemize}
    \item Unconditional model family: $\mathcal{P}$
    \item Conditional model family: 
    \[
    \tilde{\mathcal{P}} = \{ \nu_{X|C} : \forall c, \nu_{X|C}(\cdot|c) \in \mathcal{P} \}
    \]
\end{itemize}

\textbf{Divergence:} $D(\cdot\|\cdot)$
\begin{itemize}
    \item Convex in first argument
    \item Non-negative and integrable
\end{itemize}

\end{frame}

%-----------------------------------
\begin{frame}{Conditional Generative Models: Theorem}

\textbf{Assumptions:}
\begin{enumerate}
    \item \textit{Conditional distributions easier to approximate:} 
    \[
    \inf_{\nu_X} D(\mu_{X|C=c}\|\nu_X) \le \inf_{\nu_X} D(\mu_X\|\nu_X), \quad \text{a.e. }\mu_C
    \]
    \item \textit{Independent conditional choice:} $\forall c\in \mathcal{C}$, $\arg\min_{\nu_X \in \mathcal{P}} D(\mu_{X|C=c}\|\nu_X)$ exists. 
    \item \textit{Measurability \& integrability:} 
    \[
    \int_{\mathcal{C}} D(\mu_{X|C=c}\|\nu_{X|C=c}) \, \mu_C(dc) < \infty
    \]
\end{enumerate}

\vspace{2mm}
\textbf{Theorem:} 
\[
\inf_{\nu_{X|C} \in \tilde{\mathcal{P}}} \int_{\mathcal{C}} D(\mu_{X|C=c}\|\nu_{X|C=c}) \, \mu_C(dc)
\;\le\;
\inf_{\nu_X \in \mathcal{P}} D(\mu_X \|\nu_X)
\]

\vspace{2mm}
\textbf{Intuition:} splitting a complex distribution into simpler conditionals reduces the average divergence, making conditional models typically better.

\end{frame}

% 24. 参考文献
\begin{frame}{References}
    \begin{itemize}
        \item Graves, A., Srivastava, R. K., Atkinson, T., and Gomez, F. (2025). \textit{Bayesian flow networks}.
    \end{itemize}
\end{frame}

% 25. 结束页
\begin{frame}
    \centering
    \Huge \textbf{Thank You!}
\end{frame}

\end{document}